\subsubsection{Core Specification}
\label{sec:corespec}

The remainder of the radial specification is made up of the building blocks 
defined in the previous sections. Specifically, the main core lattice of fuel
assemblies is made up of the previously described fuel assemblies, separated by
the fuel assembly lattice pitch specified in Table
\ref{table_assembly_overview}. In addition, specifications for the structural
components surrounding the fuel assembly lattice are given in Table
\ref{table_baff_vess}.


\begin{table}[htbp]
  \centering
  \caption{Structural component specifications. \label{table_baff_vess}}
  
  \begin{tabularx}{\textwidth}{l C c}
    \toprule
    & & Source \\
    \midrule
    \midrule 
    Baffle Width & 2.22250 cm & \ref{num:core_baffle}\\
    Baffle Water Gap & 0.1627 cm & \ref{num:rpv}\\
    Baffle Material & \hyperlink{mat_SS304}{Stainless Steel 304} & \ref{num:rpv}\\
    
    \\
    Core Barrel \ac{IR} & 187.960 cm & \ref{num:core_barrelIR}\\
    Core Barrel \ac{OR} & 193.675 cm & \ref{num:core_barrelOR}\\
    Core Barrel Material & \hyperlink{mat_SS304}{Stainless Steel 304} & \ref{num:core_barrelmat}\\
    \\
    Neutron Shield Panel \ac{IR} & 194.840 cm & \ref{num:rpv}\\
    Neutron Shield Panel \ac{OR} & 201.630 cm & \ref{num:rpv}\\
    Neutron Shield Panel Material & \hyperlink{mat_SS304}{Stainless Steel 304} & \ref{num:rpv}\\
    Neutron Shield Panel Width & $32^{\circ}$ at the $45^{\circ}$ marks & \ref{num:rpv}\\
    \\
    Pressure Vessel Liner \ac{IR} & 219.150 cm & \ref{num:rpv}\\
    Pressure Vessel Liner \ac{OR} & 219.710 cm & \ref{num:rpv}\\
    Pressure Vessel Liner Material & \hyperlink{mat_SS304}{Stainless Steel 304} & \ref{num:catawba}\\
    \\
    Pressure Vessel \ac{IR} & 219.710 cm & \ref{num:rpv}\\
    Pressure Vessel \ac{OR} & 241.300 cm & \ref{num:rpv}\\
    Pressure Vessel Material & \hyperlink{mat_carbonsteel}{Carbon Steel 508} & \ref{num:catawba}\\
    \bottomrule
  \end{tabularx}

\end{table}

%%%%%%%%%%%%%%%%%%%%%%%%%%%%%%%%%%%%%%%%%%%%%%%%%%%%%%%%%%%%%%%%%%%%%%%%%%%%%%%%
%\FloatBarrier
\paragraph{Enrichment Zones and Burnable Absorber Positions}
\label{sec:corespec_enrba}

The initial cycle 1 fuel assembly loading pattern is shown in Figure
\ref{fig_enr_ba_pos}, including the distribution of enrichments as well as
burnable absorber locations.  The burnable absorber configurations here are
described in Section \ref{sec:ba_configs}, rotated as appropriate for core
symmetry.  A scale view of burnable absorber pins depicting these rotations is
shown in Figure \ref{fig_ba_pos_real}.

\input{specifications/core/figs/cat_enr_ba_zones} % label: fig_enr_ba_pos

\begin{figure}[htbp]
    \centering
    \includegraphics[width=6in]{specifications/core/figs/cat_ba_positions.png}
    \caption[Cycle 1 detailed burnable absorber view]{Detailed scale view of
    burnable absorber pins in cycle 1, showing proper rotations.
    \label{fig_ba_pos_real}}
\end{figure}

Figure \ref{fig_enr_ba_pos_c2} shows the shuffling pattern for cycle 2, which
includes 64 fresh assemblies and a different burnable absorber pattern.


\begin{figure}[htbp]
    \centering
    
    % these dimensions are determined in arrow_dimms.ods

    \def\scale{1.0}

    \def\latWidth{0.2673473684*\scale}
    
    \def\RPVOR{3*\scale}
    \def\rectW{0.75*\scale}
    \def\RPVIR{2.7315789474*\scale}
    \def\BarrelIR{2.3368421053*\scale}
    \def\BarrelOR{2.4078947368*\scale}
    \def\ShieldIR{2.4223787816*\scale}
    \def\ShieldOR{2.5067965189*\scale}
    \def\LinerIR{2.7246166598*\scale}

    \def\bafCIRx{0.9357157895*\scale}
    \def\bafCIRy{2.0051052632*\scale}
    \def\bafCORx{0.9633473684*\scale}
    \def\bafCORy{2.0327368421*\scale}
    \def\bafMIRx{1.7377578947*\scale}
    \def\bafMIRy{1.4704105263*\scale}
    \def\bafMORx{1.7653894737*\scale}
    \def\bafMORy{1.4980421053*\scale}
    
    \tikzset{Assembly/.style={
        inner sep=0pt,
        text width=\latWidth in,
        minimum size=\latWidth in,
        draw=black,
        align=center
        }
    }
    
    \def\tkzRPV{(0,0) circle (\RPVIR) (0,0) circle (\RPVOR)}
    \def\tkzLiner{(0,0) circle (\LinerIR) (0,0) circle (\RPVIR)}
    \def\tkzBarrel{(0,0) circle (\BarrelIR) (0,0) circle (\BarrelOR)}
    \def\tkzShields{(0,0) circle (\ShieldIR) (0,0) circle (\ShieldOR)}
    
    \def\tkzBaffCOR{(-\bafCORx, -\bafCORy) rectangle (\bafCORx, \bafCORy)}
    \def\tkzBaffCIR{(-\bafCIRx, -\bafCIRy) rectangle (\bafCIRx, \bafCIRy)} 
    \def\tkzBaffMOR{(-\bafMORx, -\bafMORy) rectangle (\bafMORx, \bafMORy)}
    \def\tkzBaffMIR{(-\bafMIRx, -\bafMIRy) rectangle (\bafMIRx, \bafMIRy) }
    \def\tkzBaffleC{ \tkzBaffCIR \tkzBaffCOR }
    \def\tkzBaffleM{ \tkzBaffMIR \tkzBaffMOR }

    \def\tkzBaffCClip{\tkzBaffCIR (-\RPVOR, -\RPVOR) rectangle (\RPVOR, \RPVOR)}
    \def\tkzBaffMClip{\tkzBaffMIR (-\RPVOR, -\RPVOR) rectangle (\RPVOR, \RPVOR)}

    \def\noenr{black!10}
    \def\lowenr{green!60!black}
    \def\highenr{orange!90}

    \scalebox{1.0}{

      \begin{tikzpicture}[x=1in,y=1in]
      
        % draw RPV, barrel, and shield panels
        
        \path[fill=black!90!white,even odd rule] \tkzRPV;
        \path[fill=black,even odd rule] \tkzLiner;
        \path[fill=black,even odd rule] \tkzBarrel;
        \begin{scope}
          \clip (0,0) -- +(61:\RPVOR) arc (61:29:\RPVOR) --
                (0,0) -- +(151:\RPVOR) arc (151:119:\RPVOR) -- 
                (0,0) -- +(241:\RPVOR) arc (241:209:\RPVOR) -- 
                (0,0) -- +(331:\RPVOR) arc (331:299:\RPVOR) -- cycle;
          \path[fill=black,even odd rule] \tkzShields;
        \end{scope}

        % draw baffle north/south
        
        \begin{scope}[even odd rule]
          \clip[rotate=90] \tkzBaffMClip;
          \path[fill=black] \tkzBaffleC;
        \end{scope}
        \begin{scope}[even odd rule]
          \clip \tkzBaffCClip;
          \clip \tkzBaffMClip;
          \path[fill=black, rotate=90] \tkzBaffleM;
        \end{scope}
        
        % draw baffle east/west
        
        \begin{scope}[rotate=90]
          \begin{scope}[even odd rule]
            \clip[rotate=90] \tkzBaffMClip;
            \path[fill=black] \tkzBaffleC;
          \end{scope}
          \begin{scope}[even odd rule]
            \clip \tkzBaffCClip;
            \clip \tkzBaffMClip;
            \path[fill=black, rotate=90] \tkzBaffleM;
          \end{scope}
        \end{scope}
        
        % draw assembly row/column headers
        
        \draw[red, thick] ($(-7*\latWidth,\RPVOR/\latWidth*\latWidth)$) node[above, anchor=south] {R} -- ($(-7*\latWidth,4*\latWidth)$);
        \draw[red, thick] ($(-6*\latWidth,\RPVOR/\latWidth*\latWidth)$) node[above, anchor=south] {P} -- ($(-6*\latWidth,6*\latWidth)$);
        \draw[red, thick] ($(-5*\latWidth,\RPVOR/\latWidth*\latWidth)$) node[above, anchor=south] {N} -- ($(-5*\latWidth,7*\latWidth)$);
        \draw[red, thick] ($(-4*\latWidth,\RPVOR/\latWidth*\latWidth)$) node[above, anchor=south] {M} -- ($(-4*\latWidth,7*\latWidth)$);
        \draw[red, thick] ($(-3*\latWidth,\RPVOR/\latWidth*\latWidth)$) node[above, anchor=south] {L} -- ($(-3*\latWidth,8*\latWidth)$);
        \draw[red, thick] ($(-2*\latWidth,\RPVOR/\latWidth*\latWidth)$) node[above, anchor=south] {K} -- ($(-2*\latWidth,8*\latWidth)$);
        \draw[red, thick] ($(-1*\latWidth,\RPVOR/\latWidth*\latWidth)$) node[above, anchor=south] {J} -- ($(-1*\latWidth,8*\latWidth)$);
        \draw[red, thick] ($(-0*\latWidth,\RPVOR/\latWidth*\latWidth)$) node[above, anchor=south] {H} -- ($(-0*\latWidth,8*\latWidth)$);
        \draw[red, thick] ($(1*\latWidth,\RPVOR/\latWidth*\latWidth)$) node[above, anchor=south] {G} -- ($(1*\latWidth,8*\latWidth)$);
        \draw[red, thick] ($(2*\latWidth,\RPVOR/\latWidth*\latWidth)$) node[above, anchor=south] {F} -- ($(2*\latWidth,8*\latWidth)$);
        \draw[red, thick] ($(3*\latWidth,\RPVOR/\latWidth*\latWidth)$) node[above, anchor=south] {E} -- ($(3*\latWidth,8*\latWidth)$);
        \draw[red, thick] ($(4*\latWidth,\RPVOR/\latWidth*\latWidth)$) node[above, anchor=south] {D} -- ($(4*\latWidth,7*\latWidth)$);
        \draw[red, thick] ($(5*\latWidth,\RPVOR/\latWidth*\latWidth)$) node[above, anchor=south] {C} -- ($(5*\latWidth,7*\latWidth)$);
        \draw[red, thick] ($(6*\latWidth,\RPVOR/\latWidth*\latWidth)$) node[above, anchor=south] {B} -- ($(6*\latWidth,6*\latWidth)$);
        \draw[red, thick] ($(7*\latWidth,\RPVOR/\latWidth*\latWidth)$) node[above, anchor=south] {A} -- ($(7*\latWidth,4*\latWidth)$);
        
        \begin{scope}[rotate=90]
          \draw[red, thick] ($(-7*\latWidth,\RPVOR/\latWidth*\latWidth)$) node[left, anchor=east] {15} -- ($(-7*\latWidth,4*\latWidth)$);
          \draw[red, thick] ($(-6*\latWidth,\RPVOR/\latWidth*\latWidth)$) node[left, anchor=east] {14} -- ($(-6*\latWidth,6*\latWidth)$);
          \draw[red, thick] ($(-5*\latWidth,\RPVOR/\latWidth*\latWidth)$) node[left, anchor=east] {13} -- ($(-5*\latWidth,7*\latWidth)$);
          \draw[red, thick] ($(-4*\latWidth,\RPVOR/\latWidth*\latWidth)$) node[left, anchor=east] {12} -- ($(-4*\latWidth,7*\latWidth)$);
          \draw[red, thick] ($(-3*\latWidth,\RPVOR/\latWidth*\latWidth)$) node[left, anchor=east] {11} -- ($(-3*\latWidth,8*\latWidth)$);
          \draw[red, thick] ($(-2*\latWidth,\RPVOR/\latWidth*\latWidth)$) node[left, anchor=east] {10} -- ($(-2*\latWidth,8*\latWidth)$);
          \draw[red, thick] ($(-1*\latWidth,\RPVOR/\latWidth*\latWidth)$) node[left, anchor=east] {9} -- ($(-1*\latWidth,8*\latWidth)$);
          \draw[red, thick] ($(-0*\latWidth,\RPVOR/\latWidth*\latWidth)$) node[left, anchor=east] {8} -- ($(-0*\latWidth,8*\latWidth)$);
          \draw[red, thick] ($(1*\latWidth,\RPVOR/\latWidth*\latWidth)$) node[left, anchor=east] {7} -- ($(1*\latWidth,8*\latWidth)$);
          \draw[red, thick] ($(2*\latWidth,\RPVOR/\latWidth*\latWidth)$) node[left, anchor=east] {6} -- ($(2*\latWidth,8*\latWidth)$);
          \draw[red, thick] ($(3*\latWidth,\RPVOR/\latWidth*\latWidth)$) node[left, anchor=east] {5} -- ($(3*\latWidth,8*\latWidth)$);
          \draw[red, thick] ($(4*\latWidth,\RPVOR/\latWidth*\latWidth)$) node[left, anchor=east] {4} -- ($(4*\latWidth,7*\latWidth)$);
          \draw[red, thick] ($(5*\latWidth,\RPVOR/\latWidth*\latWidth)$) node[left, anchor=east] {3} -- ($(5*\latWidth,7*\latWidth)$);
          \draw[red, thick] ($(6*\latWidth,\RPVOR/\latWidth*\latWidth)$) node[left, anchor=east] {2} -- ($(6*\latWidth,6*\latWidth)$);
          \draw[red, thick] ($(7*\latWidth,\RPVOR/\latWidth*\latWidth)$) node[left, anchor=east] {1} -- ($(7*\latWidth,4*\latWidth)$);
        \end{scope}
        
        % draw fuel assembly nodes
        
        \node [Assembly, fill=\noenr] at ($(-3*\latWidth,7*\latWidth)$) {\scriptsize L10}; % L1
        \node [Assembly, fill=\highenr] at ($(-2*\latWidth,7*\latWidth)$) {}; % K1
        \node [Assembly, fill=\lowenr] at ($(-1*\latWidth,7*\latWidth)$) {}; % J1
        \node [Assembly, fill=\highenr] at ($(-0*\latWidth,7*\latWidth)$) {}; % H1
        \node [Assembly, fill=\lowenr] at ($( 1*\latWidth,7*\latWidth)$) {}; % G1
        \node [Assembly, fill=\highenr] at ($( 2*\latWidth,7*\latWidth)$) {}; % F1
        \node [Assembly, fill=\noenr] at ($( 3*\latWidth,7*\latWidth)$) {\scriptsize E10}; % E1

        \node [Assembly, fill=\noenr] at ($(-5*\latWidth,6*\latWidth)$) {\scriptsize G10}; % N2
        \node [Assembly, fill=\lowenr] at ($(-4*\latWidth,6*\latWidth)$) {}; % M2
        \node [Assembly, fill=\lowenr, hyperlink node=ass_4ba_target] at ($(-3*\latWidth,6*\latWidth)$) {\small 4}; % L2
        \node [Assembly, fill=\noenr] at ($(-2*\latWidth,6*\latWidth)$) {\scriptsize L02}; % K2
        \node [Assembly, fill=\noenr] at ($(-1*\latWidth,6*\latWidth)$) {\scriptsize P12}; % J2
        \node [Assembly, fill=\noenr] at ($(-0*\latWidth,6*\latWidth)$) {\scriptsize N03}; % H2
        \node [Assembly, fill=\noenr] at ($( 1*\latWidth,6*\latWidth)$) {\scriptsize B12}; % G2
        \node [Assembly, fill=\noenr] at ($( 2*\latWidth,6*\latWidth)$) {\scriptsize E02}; % F2
        \node [Assembly, fill=\lowenr, hyperlink node=ass_4ba_target] at ($( 3*\latWidth,6*\latWidth)$) {\small 4}; % E2
        \node [Assembly, fill=\lowenr] at ($( 4*\latWidth,6*\latWidth)$) {}; % D2
        \node [Assembly, fill=\noenr] at ($( 5*\latWidth,6*\latWidth)$) {\scriptsize J10}; % C2

        \node [Assembly, fill=\noenr] at ($(-6*\latWidth,5*\latWidth)$) {\scriptsize F09}; % P3
        \node [Assembly, fill=\highenr] at ($(-5*\latWidth,5*\latWidth)$) {}; % N3
        \node [Assembly, fill=\noenr] at ($(-4*\latWidth,5*\latWidth)$) {\scriptsize N02}; % M3
        \node [Assembly, fill=\noenr] at ($(-3*\latWidth,5*\latWidth)$) {\scriptsize N10}; % L3
        \node [Assembly, fill=\lowenr, hyperlink node=ass_8ba_target] at ($(-2*\latWidth,5*\latWidth)$) {\small 8}; % K3
        \node [Assembly, fill=\noenr] at ($(-1*\latWidth,5*\latWidth)$) {\scriptsize D11}; % J3
        \node [Assembly, fill=\noenr] at ($(-0*\latWidth,5*\latWidth)$) {\scriptsize R10}; % H3
        \node [Assembly, fill=\noenr] at ($( 1*\latWidth,5*\latWidth)$) {\scriptsize M11}; % G3
        \node [Assembly, fill=\lowenr, hyperlink node=ass_8ba_target] at ($( 2*\latWidth,5*\latWidth)$) {\small 8}; % F3
        \node [Assembly, fill=\noenr] at ($( 3*\latWidth,5*\latWidth)$) {\scriptsize C10}; % E3
        \node [Assembly, fill=\noenr] at ($( 4*\latWidth,5*\latWidth)$) {\scriptsize C02}; % D3
        \node [Assembly, fill=\highenr] at ($( 5*\latWidth,5*\latWidth)$) {}; % C3
        \node [Assembly, fill=\noenr] at ($( 6*\latWidth,5*\latWidth)$) {\scriptsize K09}; % B3

        \node [Assembly, fill=\lowenr] at ($(-6*\latWidth,4*\latWidth)$) {}; % P4
        \node [Assembly, fill=\noenr] at ($(-5*\latWidth,4*\latWidth)$) {\scriptsize P03}; % N4
        \node [Assembly, fill=\noenr] at ($(-4*\latWidth,4*\latWidth)$) {\scriptsize L08}; % M4
        \node [Assembly, fill=\lowenr, hyperlink node=ass_12ba_c2_target] at ($(-3*\latWidth,4*\latWidth)$) {\small 12}; % L4
        \node [Assembly, fill=\noenr] at ($(-2*\latWidth,4*\latWidth)$) {\scriptsize M09}; % K4
        \node [Assembly, fill=\noenr] at ($(-1*\latWidth,4*\latWidth)$) {\scriptsize E15}; % J4
        \node [Assembly, fill=\noenr] at ($(-0*\latWidth,4*\latWidth)$) {\scriptsize G08}; % H4
        \node [Assembly, fill=\noenr] at ($( 1*\latWidth,4*\latWidth)$) {\scriptsize L15}; % G4
        \node [Assembly, fill=\noenr] at ($( 2*\latWidth,4*\latWidth)$) {\scriptsize D09}; % F4
        \node [Assembly, fill=\lowenr, hyperlink node=ass_12ba_c2_target] at ($( 3*\latWidth,4*\latWidth)$) {\small 12}; % E4
        \node [Assembly, fill=\noenr] at ($( 4*\latWidth,4*\latWidth)$) {\scriptsize H05}; % D4
        \node [Assembly, fill=\noenr] at ($( 5*\latWidth,4*\latWidth)$) {\scriptsize B03}; % C4
        \node [Assembly, fill=\lowenr] at ($( 6*\latWidth,4*\latWidth)$) {}; % B4

        \node [Assembly, fill=\noenr] at ($(-7*\latWidth,3*\latWidth)$) {\scriptsize F05}; % R5
        \node [Assembly, fill=\lowenr, hyperlink node=ass_4ba_target] at ($(-6*\latWidth,3*\latWidth)$) {\small 4}; % P5
        \node [Assembly, fill=\noenr] at ($(-5*\latWidth,3*\latWidth)$) {\scriptsize F03}; % N5
        \node [Assembly, fill=\lowenr, hyperlink node=ass_12ba_c2_target] at ($(-4*\latWidth,3*\latWidth)$) {\small 12}; % M5
        \node [Assembly, fill=\noenr] at ($(-3*\latWidth,3*\latWidth)$) {\scriptsize M04}; % L5
        \node [Assembly, fill=\lowenr, hyperlink node=ass_8ba_target] at ($(-2*\latWidth,3*\latWidth)$) {\small 8}; % K5
        \node [Assembly, fill=\noenr] at ($(-1*\latWidth,3*\latWidth)$) {\scriptsize M03}; % J5
        \node [Assembly, fill=\noenr] at ($(-0*\latWidth,3*\latWidth)$) {\scriptsize A10}; % H5
        \node [Assembly, fill=\noenr] at ($( 1*\latWidth,3*\latWidth)$) {\scriptsize D03}; % G5
        \node [Assembly, fill=\lowenr, hyperlink node=ass_8ba_target] at ($( 2*\latWidth,3*\latWidth)$) {\small 8}; % F5
        \node [Assembly, fill=\noenr] at ($( 3*\latWidth,3*\latWidth)$) {\scriptsize D04}; % E5
        \node [Assembly, fill=\lowenr, hyperlink node=ass_12ba_c2_target] at ($( 4*\latWidth,3*\latWidth)$) {\small 12}; % D5
        \node [Assembly, fill=\noenr] at ($( 5*\latWidth,3*\latWidth)$) {\scriptsize K03}; % C5
        \node [Assembly, fill=\lowenr, hyperlink node=ass_4ba_target] at ($( 6*\latWidth,3*\latWidth)$) {\small 4}; % B5
        \node [Assembly, fill=\noenr] at ($( 7*\latWidth,3*\latWidth)$) {\scriptsize K05}; % A5

        \node [Assembly, fill=\highenr] at ($(-7*\latWidth,2*\latWidth)$) {}; % R6
        \node [Assembly, fill=\noenr] at ($(-6*\latWidth,2*\latWidth)$) {\scriptsize P05}; % P6
        \node [Assembly, fill=\lowenr, hyperlink node=ass_8ba_target] at ($(-5*\latWidth,2*\latWidth)$) {\small 8}; % N6
        \node [Assembly, fill=\noenr] at ($(-4*\latWidth,2*\latWidth)$) {\scriptsize G04}; % M6
        \node [Assembly, fill=\lowenr, hyperlink node=ass_8ba_target] at ($(-3*\latWidth,2*\latWidth)$) {\small 8}; % L6
        \node [Assembly, fill=\noenr] at ($(-2*\latWidth,2*\latWidth)$) {\scriptsize N08}; % K6
        \node [Assembly, fill=\noenr] at ($(-1*\latWidth,2*\latWidth)$) {\scriptsize R09}; % J6
        \node [Assembly, fill=\noenr] at ($(-0*\latWidth,2*\latWidth)$) {\scriptsize G14}; % H6
        \node [Assembly, fill=\noenr] at ($( 1*\latWidth,2*\latWidth)$) {\scriptsize A09}; % G6
        \node [Assembly, fill=\noenr] at ($( 2*\latWidth,2*\latWidth)$) {\scriptsize H03}; % F6
        \node [Assembly, fill=\lowenr, hyperlink node=ass_8ba_target] at ($( 3*\latWidth,2*\latWidth)$) {\small 8}; % E6
        \node [Assembly, fill=\noenr] at ($( 4*\latWidth,2*\latWidth)$) {\scriptsize J04}; % D6
        \node [Assembly, fill=\lowenr, hyperlink node=ass_8ba_target] at ($( 5*\latWidth,2*\latWidth)$) {\small 8}; % C6
        \node [Assembly, fill=\noenr] at ($( 6*\latWidth,2*\latWidth)$) {\scriptsize B05}; % B6
        \node [Assembly, fill=\highenr] at ($( 7*\latWidth,2*\latWidth)$) {}; % A6

        \node [Assembly, fill=\lowenr] at ($(-7*\latWidth,1*\latWidth)$) {}; % R7
        \node [Assembly, fill=\noenr] at ($(-6*\latWidth,1*\latWidth)$) {\scriptsize D02}; % P7
        \node [Assembly, fill=\noenr] at ($(-5*\latWidth,1*\latWidth)$) {\scriptsize E12}; % N7
        \node [Assembly, fill=\noenr] at ($(-4*\latWidth,1*\latWidth)$) {\scriptsize A11}; % M7
        \node [Assembly, fill=\noenr] at ($(-3*\latWidth,1*\latWidth)$) {\scriptsize N04}; % L7
        \node [Assembly, fill=\noenr] at ($(-2*\latWidth,1*\latWidth)$) {\scriptsize G01}; % K7
        \node [Assembly, fill=\noenr] at ($(-1*\latWidth,1*\latWidth)$) {\scriptsize B09}; % J7
        \node [Assembly, fill=\noenr] at ($(-0*\latWidth,1*\latWidth)$) {\scriptsize H15}; % H7
        \node [Assembly, fill=\noenr] at ($( 1*\latWidth,1*\latWidth)$) {\scriptsize J14}; % G7
        \node [Assembly, fill=\noenr] at ($( 2*\latWidth,1*\latWidth)$) {\scriptsize J01}; % F7
        \node [Assembly, fill=\noenr] at ($( 3*\latWidth,1*\latWidth)$) {\scriptsize C04}; % E7
        \node [Assembly, fill=\noenr] at ($( 4*\latWidth,1*\latWidth)$) {\scriptsize R11}; % D7
        \node [Assembly, fill=\noenr] at ($( 5*\latWidth,1*\latWidth)$) {\scriptsize L12}; % C7
        \node [Assembly, fill=\noenr] at ($( 6*\latWidth,1*\latWidth)$) {\scriptsize M02}; % B7
        \node [Assembly, fill=\lowenr] at ($( 7*\latWidth,1*\latWidth)$) {}; % A7

        \node [Assembly, fill=\highenr] at ($(-7*\latWidth,0*\latWidth)$) {}; % R8
        \node [Assembly, fill=\noenr] at ($(-6*\latWidth,0*\latWidth)$) {\scriptsize N13}; % P8
        \node [Assembly, fill=\noenr] at ($(-5*\latWidth,0*\latWidth)$) {\scriptsize F15}; % N8
        \node [Assembly, fill=\noenr] at ($(-4*\latWidth,0*\latWidth)$) {\scriptsize H07}; % M8
        \node [Assembly, fill=\noenr] at ($(-3*\latWidth,0*\latWidth)$) {\scriptsize F01}; % L8
        \node [Assembly, fill=\noenr] at ($(-2*\latWidth,0*\latWidth)$) {\scriptsize B07}; % K8
        \node [Assembly, fill=\noenr] at ($(-1*\latWidth,0*\latWidth)$) {\scriptsize A08}; % J8
        \node [Assembly, fill=\noenr] at ($(-0*\latWidth,0*\latWidth)$) {\scriptsize F14}; % H8
        \node [Assembly, fill=\noenr] at ($( 1*\latWidth,0*\latWidth)$) {\scriptsize R08}; % G8
        \node [Assembly, fill=\noenr] at ($( 2*\latWidth,0*\latWidth)$) {\scriptsize P09}; % F8
        \node [Assembly, fill=\noenr] at ($( 3*\latWidth,0*\latWidth)$) {\scriptsize K15}; % E8
        \node [Assembly, fill=\noenr] at ($( 4*\latWidth,0*\latWidth)$) {\scriptsize H09}; % D8
        \node [Assembly, fill=\noenr] at ($( 5*\latWidth,0*\latWidth)$) {\scriptsize K01}; % C8
        \node [Assembly, fill=\noenr] at ($( 6*\latWidth,0*\latWidth)$) {\scriptsize C03}; % B8
        \node [Assembly, fill=\highenr] at ($( 7*\latWidth,0*\latWidth)$) {}; % A8

        \node [Assembly, fill=\lowenr] at ($(-7*\latWidth,-1*\latWidth)$) {}; % R9
        \node [Assembly, fill=\noenr] at ($(-6*\latWidth,-1*\latWidth)$) {\scriptsize D14}; % P9
        \node [Assembly, fill=\noenr] at ($(-5*\latWidth,-1*\latWidth)$) {\scriptsize E04}; % N9
        \node [Assembly, fill=\noenr] at ($(-4*\latWidth,-1*\latWidth)$) {\scriptsize A05}; % M9
        \node [Assembly, fill=\noenr] at ($(-3*\latWidth,-1*\latWidth)$) {\scriptsize N12}; % L9
        \node [Assembly, fill=\noenr] at ($(-2*\latWidth,-1*\latWidth)$) {\scriptsize G15}; % K9
        \node [Assembly, fill=\noenr] at ($(-1*\latWidth,-1*\latWidth)$) {\scriptsize G02}; % J9
        \node [Assembly, fill=\noenr] at ($(-0*\latWidth,-1*\latWidth)$) {\scriptsize H01}; % H9
        \node [Assembly, fill=\noenr] at ($( 1*\latWidth,-1*\latWidth)$) {\scriptsize P07}; % G9
        \node [Assembly, fill=\noenr] at ($( 2*\latWidth,-1*\latWidth)$) {\scriptsize J15}; % F9
        \node [Assembly, fill=\noenr] at ($( 3*\latWidth,-1*\latWidth)$) {\scriptsize C12}; % E9
        \node [Assembly, fill=\noenr] at ($( 4*\latWidth,-1*\latWidth)$) {\scriptsize R05}; % D9
        \node [Assembly, fill=\noenr] at ($( 5*\latWidth,-1*\latWidth)$) {\scriptsize L04}; % C9
        \node [Assembly, fill=\noenr] at ($( 6*\latWidth,-1*\latWidth)$) {\scriptsize M14}; % B9
        \node [Assembly, fill=\lowenr] at ($( 7*\latWidth,-1*\latWidth)$) {}; % A9

        \node [Assembly, fill=\highenr] at ($(-7*\latWidth,-2*\latWidth)$) {}; % R10
        \node [Assembly, fill=\noenr] at ($(-6*\latWidth,-2*\latWidth)$) {\scriptsize P11}; % P10
        \node [Assembly, fill=\lowenr, hyperlink node=ass_8ba_target] at ($(-5*\latWidth,-2*\latWidth)$) {\small 8}; % N10
        \node [Assembly, fill=\noenr] at ($(-4*\latWidth,-2*\latWidth)$) {\scriptsize G12}; % M10
        \node [Assembly, fill=\lowenr, hyperlink node=ass_8ba_target] at ($(-3*\latWidth,-2*\latWidth)$) {\small 8}; % L10
        \node [Assembly, fill=\noenr] at ($(-2*\latWidth,-2*\latWidth)$) {\scriptsize H13}; % K10
        \node [Assembly, fill=\noenr] at ($(-1*\latWidth,-2*\latWidth)$) {\scriptsize R07}; % J10
        \node [Assembly, fill=\noenr] at ($(-0*\latWidth,-2*\latWidth)$) {\scriptsize J02}; % H10
        \node [Assembly, fill=\noenr] at ($( 1*\latWidth,-2*\latWidth)$) {\scriptsize A07}; % G10
        \node [Assembly, fill=\noenr] at ($( 2*\latWidth,-2*\latWidth)$) {\scriptsize C08}; % F10
        \node [Assembly, fill=\lowenr, hyperlink node=ass_8ba_target] at ($( 3*\latWidth,-2*\latWidth)$) {\small 8}; % E10
        \node [Assembly, fill=\noenr] at ($( 4*\latWidth,-2*\latWidth)$) {\scriptsize J12}; % D10
        \node [Assembly, fill=\lowenr, hyperlink node=ass_8ba_target] at ($( 5*\latWidth,-2*\latWidth)$) {\small 8}; % C10
        \node [Assembly, fill=\noenr] at ($( 6*\latWidth,-2*\latWidth)$) {\scriptsize B11}; % B10
        \node [Assembly, fill=\highenr] at ($( 7*\latWidth,-2*\latWidth)$) {}; % A10

        \node [Assembly, fill=\noenr] at ($(-7*\latWidth,-3*\latWidth)$) {\scriptsize F11}; % R11
        \node [Assembly, fill=\lowenr, hyperlink node=ass_4ba_target] at ($(-6*\latWidth,-3*\latWidth)$) {\small 4}; % P11
        \node [Assembly, fill=\noenr] at ($(-5*\latWidth,-3*\latWidth)$) {\scriptsize F13}; % N11
        \node [Assembly, fill=\lowenr, hyperlink node=ass_12ba_c2_target] at ($(-4*\latWidth,-3*\latWidth)$) {\small 12}; % M11
        \node [Assembly, fill=\noenr] at ($(-3*\latWidth,-3*\latWidth)$) {\scriptsize M12}; % L11
        \node [Assembly, fill=\lowenr, hyperlink node=ass_8ba_target] at ($(-2*\latWidth,-3*\latWidth)$) {\small 8}; % K11
        \node [Assembly, fill=\noenr] at ($(-1*\latWidth,-3*\latWidth)$) {\scriptsize M13}; % J11
        \node [Assembly, fill=\noenr] at ($(-0*\latWidth,-3*\latWidth)$) {\scriptsize R06}; % H11
        \node [Assembly, fill=\noenr] at ($( 1*\latWidth,-3*\latWidth)$) {\scriptsize D13}; % G11
        \node [Assembly, fill=\lowenr, hyperlink node=ass_8ba_target] at ($( 2*\latWidth,-3*\latWidth)$) {\small 8}; % F11
        \node [Assembly, fill=\noenr] at ($( 3*\latWidth,-3*\latWidth)$) {\scriptsize D12}; % E11
        \node [Assembly, fill=\lowenr, hyperlink node=ass_12ba_c2_target] at ($( 4*\latWidth,-3*\latWidth)$) {\small 12}; % D11
        \node [Assembly, fill=\noenr] at ($( 5*\latWidth,-3*\latWidth)$) {\scriptsize K13}; % C11
        \node [Assembly, fill=\lowenr, hyperlink node=ass_4ba_target] at ($( 6*\latWidth,-3*\latWidth)$) {\small 4}; % B11
        \node [Assembly, fill=\noenr] at ($( 7*\latWidth,-3*\latWidth)$) {\scriptsize K11}; % A11

        \node [Assembly, fill=\lowenr] at ($(-6*\latWidth,-4*\latWidth)$) {}; % P12
        \node [Assembly, fill=\noenr] at ($(-5*\latWidth,-4*\latWidth)$) {\scriptsize P13}; % N12
        \node [Assembly, fill=\noenr] at ($(-4*\latWidth,-4*\latWidth)$) {\scriptsize H11}; % M12
        \node [Assembly, fill=\lowenr, hyperlink node=ass_12ba_c2_target] at ($(-3*\latWidth,-4*\latWidth)$) {\small 12}; % L12
        \node [Assembly, fill=\noenr] at ($(-2*\latWidth,-4*\latWidth)$) {\scriptsize M07}; % K12
        \node [Assembly, fill=\noenr] at ($(-1*\latWidth,-4*\latWidth)$) {\scriptsize E01}; % J12
        \node [Assembly, fill=\noenr] at ($(-0*\latWidth,-4*\latWidth)$) {\scriptsize J08}; % H12
        \node [Assembly, fill=\noenr] at ($( 1*\latWidth,-4*\latWidth)$) {\scriptsize L01}; % G12
        \node [Assembly, fill=\noenr] at ($( 2*\latWidth,-4*\latWidth)$) {\scriptsize D07}; % F12
        \node [Assembly, fill=\lowenr, hyperlink node=ass_12ba_c2_target] at ($( 3*\latWidth,-4*\latWidth)$) {\small 12}; % E12
        \node [Assembly, fill=\noenr] at ($( 4*\latWidth,-4*\latWidth)$) {\scriptsize E08}; % D12
        \node [Assembly, fill=\noenr] at ($( 5*\latWidth,-4*\latWidth)$) {\scriptsize B13}; % C12
        \node [Assembly, fill=\lowenr] at ($( 6*\latWidth,-4*\latWidth)$) {}; % B12

        \node [Assembly, fill=\noenr] at ($(-6*\latWidth,-5*\latWidth)$) {\scriptsize F07}; % P13
        \node [Assembly, fill=\highenr] at ($(-5*\latWidth,-5*\latWidth)$) {}; % N13
        \node [Assembly, fill=\noenr] at ($(-4*\latWidth,-5*\latWidth)$) {\scriptsize N14}; % M13
        \node [Assembly, fill=\noenr] at ($(-3*\latWidth,-5*\latWidth)$) {\scriptsize N06}; % L13
        \node [Assembly, fill=\lowenr, hyperlink node=ass_8ba_target] at ($(-2*\latWidth,-5*\latWidth)$) {\small 8}; % K13
        \node [Assembly, fill=\noenr] at ($(-1*\latWidth,-5*\latWidth)$) {\scriptsize D05}; % J13
        \node [Assembly, fill=\noenr] at ($(-0*\latWidth,-5*\latWidth)$) {\scriptsize A06}; % H13
        \node [Assembly, fill=\noenr] at ($( 1*\latWidth,-5*\latWidth)$) {\scriptsize M05}; % G13
        \node [Assembly, fill=\lowenr, hyperlink node=ass_8ba_target] at ($( 2*\latWidth,-5*\latWidth)$) {\small 8}; % F13
        \node [Assembly, fill=\noenr] at ($( 3*\latWidth,-5*\latWidth)$) {\scriptsize C06}; % E13
        \node [Assembly, fill=\noenr] at ($( 4*\latWidth,-5*\latWidth)$) {\scriptsize C14}; % D13
        \node [Assembly, fill=\highenr] at ($( 5*\latWidth,-5*\latWidth)$) {}; % C13
        \node [Assembly, fill=\noenr] at ($( 6*\latWidth,-5*\latWidth)$) {\scriptsize K07}; % B13

        \node [Assembly, fill=\noenr] at ($(-5*\latWidth,-6*\latWidth)$) {\scriptsize G06}; % N14
        \node [Assembly, fill=\lowenr] at ($(-4*\latWidth,-6*\latWidth)$) {}; % M14
        \node [Assembly, fill=\lowenr, hyperlink node=ass_4ba_target] at ($(-3*\latWidth,-6*\latWidth)$) {\small 4}; % L14
        \node [Assembly, fill=\noenr] at ($(-2*\latWidth,-6*\latWidth)$) {\scriptsize L14}; % K14
        \node [Assembly, fill=\noenr] at ($(-1*\latWidth,-6*\latWidth)$) {\scriptsize P04}; % J14
        \node [Assembly, fill=\noenr] at ($(-0*\latWidth,-6*\latWidth)$) {\scriptsize C13}; % H14
        \node [Assembly, fill=\noenr] at ($( 1*\latWidth,-6*\latWidth)$) {\scriptsize B04}; % G14
        \node [Assembly, fill=\noenr] at ($( 2*\latWidth,-6*\latWidth)$) {\scriptsize E14}; % F14
        \node [Assembly, fill=\lowenr, hyperlink node=ass_4ba_target] at ($( 3*\latWidth,-6*\latWidth)$) {\small 4}; % E14
        \node [Assembly, fill=\lowenr] at ($( 4*\latWidth,-6*\latWidth)$) {}; % D14
        \node [Assembly, fill=\noenr] at ($( 5*\latWidth,-6*\latWidth)$) {\scriptsize J06}; % C14

        \node [Assembly, fill=\noenr] at ($(-3*\latWidth,-7*\latWidth)$) {\scriptsize L06}; % L15
        \node [Assembly, fill=\highenr] at ($(-2*\latWidth,-7*\latWidth)$) {}; % K15
        \node [Assembly, fill=\lowenr] at ($(-1*\latWidth,-7*\latWidth)$) {}; % J15
        \node [Assembly, fill=\highenr] at ($(-0*\latWidth,-7*\latWidth)$) {}; % H15
        \node [Assembly, fill=\lowenr] at ($( 1*\latWidth,-7*\latWidth)$) {}; % G15
        \node [Assembly, fill=\highenr] at ($( 2*\latWidth,-7*\latWidth)$) {}; % F15
        \node [Assembly, fill=\noenr] at ($( 3*\latWidth,-7*\latWidth)$) {\scriptsize E06}; % E15
        
      \end{tikzpicture}
    }
    
    % make the legend
    \begin{tikzpicture}
      \matrix [matrix of nodes]
          {
              \node [Assembly, fill=\lowenr, hyperlink node=mat_fuel32] at (0,0) {}; & \hyperref[mat_fuel32]{Fresh 3.2 w/o U235}~~~ & \node [Assembly, fill=\highenr, hyperlink node=mat_fuel34] at (0,0) {}; & \hyperref[mat_fuel34]{Fresh 3.4 w/o U235}~~~ \\
              \node [Assembly, fill=\noenr] at (0,0) {}; & Shuffled Assembly~~~ & ~~~ & ~~~\\
          };
    \end{tikzpicture}

    \caption[Cycle 2 shuffling pattern and burnable absorber positions]{Cycle 2 shuffling pattern, burnable absorber positions, and enrichment loading pattern of fresh assemblies. Sources: \ref{num:assycore}, \ref{num:c2shuffle} \label{fig_enr_ba_pos_c2}}
\end{figure}

 % label: fig_enr_ba_pos_c2

%%%%%%%%%%%%%%%%%%%%%%%%%%%%%%%%%%%%%%%%%%%%%%%%%%%%%%%%%%%%%%%%%%%%%%%%%%%%%%%%
%\FloatBarrier
\paragraph{Control Rod Bank Positions}

Each of the four control rod banks - specified by the identifiers A, B, C, and D
- are made up of several control rod clusters in multiple fuel assemblies. In
control rod clusters, every guide tube is filled with the control rod pincell
described in section \ref{sec:pintypes}, with the exception of the center
tube. Each of the clusters in a given control rod bank move together.

In addition to the control rod banks, 5 shutdown banks of control rod clusters
are included above the core - specified by $\mathrm{S}_\mathrm{A}$,
$\mathrm{S}_\mathrm{B}$, $\mathrm{S}_\mathrm{C}$, $\mathrm{S}_\mathrm{D}$, and
$\mathrm{S}_\mathrm{E}$. These clusters are not used in normal operation,
however, their reactivity worth was measured and reported in Table
\ref{tbl:meas_c1phys}.

Figure \ref{fig_cr_pos} shows the radial locations of control rod clusters
belonging to each control rod and shutdown bank. The axial specifications of
each are described later in Section \ref{sec:axial_cr}.

The positions of control rod banks do not change between cycle 1 and cycle 2.


\begin{figure}[htbp]
    \centering
    
    % these dimensions are determined in arrow_dimms.ods

    \def\scale{1.0}

    \def\latWidth{0.2673473684*\scale}
    
    \def\RPVOR{3*\scale}
    \def\rectW{0.75*\scale}
    \def\RPVIR{2.7315789474*\scale}
    \def\BarrelIR{2.3368421053*\scale}
    \def\BarrelOR{2.4078947368*\scale}
    \def\ShieldIR{2.4223787816*\scale}
    \def\ShieldOR{2.5067965189*\scale}
    \def\LinerIR{2.7246166598*\scale}

    \def\bafCIRx{0.9357157895*\scale}
    \def\bafCIRy{2.0051052632*\scale}
    \def\bafCORx{0.9633473684*\scale}
    \def\bafCORy{2.0327368421*\scale}
    \def\bafMIRx{1.7377578947*\scale}
    \def\bafMIRy{1.4704105263*\scale}
    \def\bafMORx{1.7653894737*\scale}
    \def\bafMORy{1.4980421053*\scale}
    
    \tikzset{Assembly/.style={
        inner sep=0pt,
        text width=\latWidth in,
        minimum size=\latWidth in,
        draw=black,
        align=center
        }
    }
    
    \def\tkzRPV{(0,0) circle (\RPVIR) (0,0) circle (\RPVOR)}
    \def\tkzLiner{(0,0) circle (\LinerIR) (0,0) circle (\RPVIR)}
    \def\tkzBarrel{(0,0) circle (\BarrelIR) (0,0) circle (\BarrelOR)}
    \def\tkzShields{(0,0) circle (\ShieldIR) (0,0) circle (\ShieldOR)}
    
    \def\tkzBaffCOR{(-\bafCORx, -\bafCORy) rectangle (\bafCORx, \bafCORy)}
    \def\tkzBaffCIR{(-\bafCIRx, -\bafCIRy) rectangle (\bafCIRx, \bafCIRy)} 
    \def\tkzBaffMOR{(-\bafMORx, -\bafMORy) rectangle (\bafMORx, \bafMORy)}
    \def\tkzBaffMIR{(-\bafMIRx, -\bafMIRy) rectangle (\bafMIRx, \bafMIRy) }
    \def\tkzBaffleC{ \tkzBaffCIR \tkzBaffCOR }
    \def\tkzBaffleM{ \tkzBaffMIR \tkzBaffMOR }

    \def\tkzBaffCClip{\tkzBaffCIR (-\RPVOR, -\RPVOR) rectangle (\RPVOR, \RPVOR)}
    \def\tkzBaffMClip{\tkzBaffMIR (-\RPVOR, -\RPVOR) rectangle (\RPVOR, \RPVOR)}

    \def\highenr{blue!50}
    \def\midenr{yellow!50}
    \def\lowenr{red!50}

    \scalebox{1.0}{

      \begin{tikzpicture}[x=1in,y=1in]
      
        % draw RPV, barrel, liner and shield panels
        
        \path[fill=black!90!white,even odd rule] \tkzRPV;
        \path[fill=black,even odd rule] \tkzLiner;
        \path[fill=black,even odd rule] \tkzBarrel;
        \begin{scope}
          \clip (0,0) -- +(61:\RPVOR) arc (61:29:\RPVOR) --
                (0,0) -- +(151:\RPVOR) arc (151:119:\RPVOR) -- 
                (0,0) -- +(241:\RPVOR) arc (241:209:\RPVOR) -- 
                (0,0) -- +(331:\RPVOR) arc (331:299:\RPVOR) -- cycle;
          \path[fill=black,even odd rule] \tkzShields;
        \end{scope}

        % draw baffle north/south
        
        \begin{scope}[even odd rule]
          \clip[rotate=90] \tkzBaffMClip;
          \path[fill=black] \tkzBaffleC;
        \end{scope}
        \begin{scope}[even odd rule]
          \clip \tkzBaffCClip;
          \clip \tkzBaffMClip;
          \path[fill=black, rotate=90] \tkzBaffleM;
        \end{scope}
        
        % draw baffle east/west
        
        \begin{scope}[rotate=90]
          \begin{scope}[even odd rule]
            \clip[rotate=90] \tkzBaffMClip;
            \path[fill=black] \tkzBaffleC;
          \end{scope}
          \begin{scope}[even odd rule]
            \clip \tkzBaffCClip;
            \clip \tkzBaffMClip;
            \path[fill=black, rotate=90] \tkzBaffleM;
          \end{scope}
        \end{scope}
        
        % draw assembly row/column headers
        
        \draw[red, thick] ($(-7*\latWidth,\RPVOR/\latWidth*\latWidth)$) node[above, anchor=south] {R} -- ($(-7*\latWidth,4*\latWidth)$);
        \draw[red, thick] ($(-6*\latWidth,\RPVOR/\latWidth*\latWidth)$) node[above, anchor=south] {P} -- ($(-6*\latWidth,6*\latWidth)$);
        \draw[red, thick] ($(-5*\latWidth,\RPVOR/\latWidth*\latWidth)$) node[above, anchor=south] {N} -- ($(-5*\latWidth,7*\latWidth)$);
        \draw[red, thick] ($(-4*\latWidth,\RPVOR/\latWidth*\latWidth)$) node[above, anchor=south] {M} -- ($(-4*\latWidth,7*\latWidth)$);
        \draw[red, thick] ($(-3*\latWidth,\RPVOR/\latWidth*\latWidth)$) node[above, anchor=south] {L} -- ($(-3*\latWidth,8*\latWidth)$);
        \draw[red, thick] ($(-2*\latWidth,\RPVOR/\latWidth*\latWidth)$) node[above, anchor=south] {K} -- ($(-2*\latWidth,8*\latWidth)$);
        \draw[red, thick] ($(-1*\latWidth,\RPVOR/\latWidth*\latWidth)$) node[above, anchor=south] {J} -- ($(-1*\latWidth,8*\latWidth)$);
        \draw[red, thick] ($(-0*\latWidth,\RPVOR/\latWidth*\latWidth)$) node[above, anchor=south] {H} -- ($(-0*\latWidth,8*\latWidth)$);
        \draw[red, thick] ($(1*\latWidth,\RPVOR/\latWidth*\latWidth)$) node[above, anchor=south] {G} -- ($(1*\latWidth,8*\latWidth)$);
        \draw[red, thick] ($(2*\latWidth,\RPVOR/\latWidth*\latWidth)$) node[above, anchor=south] {F} -- ($(2*\latWidth,8*\latWidth)$);
        \draw[red, thick] ($(3*\latWidth,\RPVOR/\latWidth*\latWidth)$) node[above, anchor=south] {E} -- ($(3*\latWidth,8*\latWidth)$);
        \draw[red, thick] ($(4*\latWidth,\RPVOR/\latWidth*\latWidth)$) node[above, anchor=south] {D} -- ($(4*\latWidth,7*\latWidth)$);
        \draw[red, thick] ($(5*\latWidth,\RPVOR/\latWidth*\latWidth)$) node[above, anchor=south] {C} -- ($(5*\latWidth,7*\latWidth)$);
        \draw[red, thick] ($(6*\latWidth,\RPVOR/\latWidth*\latWidth)$) node[above, anchor=south] {B} -- ($(6*\latWidth,6*\latWidth)$);
        \draw[red, thick] ($(7*\latWidth,\RPVOR/\latWidth*\latWidth)$) node[above, anchor=south] {A} -- ($(7*\latWidth,4*\latWidth)$);
        
        \begin{scope}[rotate=90]
          \draw[red, thick] ($(-7*\latWidth,\RPVOR/\latWidth*\latWidth)$) node[left, anchor=east] {15} -- ($(-7*\latWidth,4*\latWidth)$);
          \draw[red, thick] ($(-6*\latWidth,\RPVOR/\latWidth*\latWidth)$) node[left, anchor=east] {14} -- ($(-6*\latWidth,6*\latWidth)$);
          \draw[red, thick] ($(-5*\latWidth,\RPVOR/\latWidth*\latWidth)$) node[left, anchor=east] {13} -- ($(-5*\latWidth,7*\latWidth)$);
          \draw[red, thick] ($(-4*\latWidth,\RPVOR/\latWidth*\latWidth)$) node[left, anchor=east] {12} -- ($(-4*\latWidth,7*\latWidth)$);
          \draw[red, thick] ($(-3*\latWidth,\RPVOR/\latWidth*\latWidth)$) node[left, anchor=east] {11} -- ($(-3*\latWidth,8*\latWidth)$);
          \draw[red, thick] ($(-2*\latWidth,\RPVOR/\latWidth*\latWidth)$) node[left, anchor=east] {10} -- ($(-2*\latWidth,8*\latWidth)$);
          \draw[red, thick] ($(-1*\latWidth,\RPVOR/\latWidth*\latWidth)$) node[left, anchor=east] {9} -- ($(-1*\latWidth,8*\latWidth)$);
          \draw[red, thick] ($(-0*\latWidth,\RPVOR/\latWidth*\latWidth)$) node[left, anchor=east] {8} -- ($(-0*\latWidth,8*\latWidth)$);
          \draw[red, thick] ($(1*\latWidth,\RPVOR/\latWidth*\latWidth)$) node[left, anchor=east] {7} -- ($(1*\latWidth,8*\latWidth)$);
          \draw[red, thick] ($(2*\latWidth,\RPVOR/\latWidth*\latWidth)$) node[left, anchor=east] {6} -- ($(2*\latWidth,8*\latWidth)$);
          \draw[red, thick] ($(3*\latWidth,\RPVOR/\latWidth*\latWidth)$) node[left, anchor=east] {5} -- ($(3*\latWidth,8*\latWidth)$);
          \draw[red, thick] ($(4*\latWidth,\RPVOR/\latWidth*\latWidth)$) node[left, anchor=east] {4} -- ($(4*\latWidth,7*\latWidth)$);
          \draw[red, thick] ($(5*\latWidth,\RPVOR/\latWidth*\latWidth)$) node[left, anchor=east] {3} -- ($(5*\latWidth,7*\latWidth)$);
          \draw[red, thick] ($(6*\latWidth,\RPVOR/\latWidth*\latWidth)$) node[left, anchor=east] {2} -- ($(6*\latWidth,6*\latWidth)$);
          \draw[red, thick] ($(7*\latWidth,\RPVOR/\latWidth*\latWidth)$) node[left, anchor=east] {1} -- ($(7*\latWidth,4*\latWidth)$);
        \end{scope}
        
        % draw fuel assembly nodes
        
        \node [Assembly, fill=\highenr] at ($(-3*\latWidth,7*\latWidth)$) {}; % L1
        \node [Assembly, fill=\highenr] at ($(-2*\latWidth,7*\latWidth)$) {}; % K1
        \node [Assembly, fill=\highenr] at ($(-1*\latWidth,7*\latWidth)$) {}; % J1
        \node [Assembly, fill=\highenr] at ($(-0*\latWidth,7*\latWidth)$) {}; % H1
        \node [Assembly, fill=\highenr] at ($( 1*\latWidth,7*\latWidth)$) {}; % G1
        \node [Assembly, fill=\highenr] at ($( 2*\latWidth,7*\latWidth)$) {}; % F1
        \node [Assembly, fill=\highenr] at ($( 3*\latWidth,7*\latWidth)$) {}; % E1

        \node [Assembly, fill=\highenr] at ($(-5*\latWidth,6*\latWidth)$) {}; % N2
        \node [Assembly, fill=\highenr] at ($(-4*\latWidth,6*\latWidth)$) {$\mathrm{S}_\mathrm{A}$}; % M2
        \node [Assembly, fill=\highenr] at ($(-3*\latWidth,6*\latWidth)$) {}; % L2
        \node [Assembly, fill=\lowenr] at ($(-2*\latWidth,6*\latWidth)$) {$\mathrm{B}$}; % K2
        \node [Assembly, fill=\highenr] at ($(-1*\latWidth,6*\latWidth)$) {}; % J2
        \node [Assembly, fill=\lowenr] at ($(-0*\latWidth,6*\latWidth)$) {$\mathrm{C}$}; % H2
        \node [Assembly, fill=\highenr] at ($( 1*\latWidth,6*\latWidth)$) {}; % G2
        \node [Assembly, fill=\lowenr] at ($( 2*\latWidth,6*\latWidth)$) {$\mathrm{B}$}; % F2
        \node [Assembly, fill=\highenr] at ($( 3*\latWidth,6*\latWidth)$) {}; % E2
        \node [Assembly, fill=\highenr] at ($( 4*\latWidth,6*\latWidth)$) {$\mathrm{S}_\mathrm{A}$}; % D2
        \node [Assembly, fill=\highenr] at ($( 5*\latWidth,6*\latWidth)$) {}; % C2

        \node [Assembly, fill=\highenr] at ($(-6*\latWidth,5*\latWidth)$) {}; % P3
        \node [Assembly, fill=\highenr] at ($(-5*\latWidth,5*\latWidth)$) {}; % N3
        \node [Assembly, fill=\midenr] at ($(-4*\latWidth,5*\latWidth)$) {}; % M3
        \node [Assembly, fill=\lowenr] at ($(-3*\latWidth,5*\latWidth)$) {$\mathrm{S}_\mathrm{D}$}; % L3
        \node [Assembly, fill=\midenr] at ($(-2*\latWidth,5*\latWidth)$) {}; % K3
        \node [Assembly, fill=\lowenr] at ($(-1*\latWidth,5*\latWidth)$) {$\mathrm{S}_\mathrm{B}$}; % J3
        \node [Assembly, fill=\midenr] at ($(-0*\latWidth,5*\latWidth)$) {}; % H3
        \node [Assembly, fill=\lowenr] at ($( 1*\latWidth,5*\latWidth)$) {$\mathrm{S}_\mathrm{B}$}; % G3
        \node [Assembly, fill=\midenr] at ($( 2*\latWidth,5*\latWidth)$) {}; % F3
        \node [Assembly, fill=\lowenr] at ($( 3*\latWidth,5*\latWidth)$) {$\mathrm{S}_\mathrm{C}$}; % E3
        \node [Assembly, fill=\midenr] at ($( 4*\latWidth,5*\latWidth)$) {}; % D3
        \node [Assembly, fill=\highenr] at ($( 5*\latWidth,5*\latWidth)$) {}; % C3
        \node [Assembly, fill=\highenr] at ($( 6*\latWidth,5*\latWidth)$) {}; % B3

        \node [Assembly, fill=\highenr] at ($(-6*\latWidth,4*\latWidth)$) {$\mathrm{S}_\mathrm{A}$}; % P4
        \node [Assembly, fill=\midenr] at ($(-5*\latWidth,4*\latWidth)$) {}; % N4
        \node [Assembly, fill=\midenr] at ($(-4*\latWidth,4*\latWidth)$) {$\mathrm{D}$}; % M4
        \node [Assembly, fill=\midenr] at ($(-3*\latWidth,4*\latWidth)$) {}; % L4
        \node [Assembly, fill=\lowenr] at ($(-2*\latWidth,4*\latWidth)$) {}; % K4
        \node [Assembly, fill=\midenr] at ($(-1*\latWidth,4*\latWidth)$) {}; % J4
        \node [Assembly, fill=\lowenr] at ($(-0*\latWidth,4*\latWidth)$) {$\mathrm{S}_\mathrm{E}$}; % H4
        \node [Assembly, fill=\midenr] at ($( 1*\latWidth,4*\latWidth)$) {}; % G4
        \node [Assembly, fill=\lowenr] at ($( 2*\latWidth,4*\latWidth)$) {}; % F4
        \node [Assembly, fill=\midenr] at ($( 3*\latWidth,4*\latWidth)$) {}; % E4
        \node [Assembly, fill=\midenr] at ($( 4*\latWidth,4*\latWidth)$) {$\mathrm{D}$}; % D4
        \node [Assembly, fill=\midenr] at ($( 5*\latWidth,4*\latWidth)$) {}; % C4
        \node [Assembly, fill=\highenr] at ($( 6*\latWidth,4*\latWidth)$) {$\mathrm{S}_\mathrm{A}$}; % B4

        \node [Assembly, fill=\highenr] at ($(-7*\latWidth,3*\latWidth)$) {}; % R5
        \node [Assembly, fill=\highenr] at ($(-6*\latWidth,3*\latWidth)$) {}; % P5
        \node [Assembly, fill=\lowenr] at ($(-5*\latWidth,3*\latWidth)$) {$\mathrm{S}_\mathrm{C}$}; % N5
        \node [Assembly, fill=\midenr] at ($(-4*\latWidth,3*\latWidth)$) {}; % M5
        \node [Assembly, fill=\lowenr] at ($(-3*\latWidth,3*\latWidth)$) {}; % L5
        \node [Assembly, fill=\midenr] at ($(-2*\latWidth,3*\latWidth)$) {}; % K5
        \node [Assembly, fill=\lowenr] at ($(-1*\latWidth,3*\latWidth)$) {}; % J5
        \node [Assembly, fill=\midenr] at ($(-0*\latWidth,3*\latWidth)$) {}; % H5
        \node [Assembly, fill=\lowenr] at ($( 1*\latWidth,3*\latWidth)$) {}; % G5
        \node [Assembly, fill=\midenr] at ($( 2*\latWidth,3*\latWidth)$) {}; % F5
        \node [Assembly, fill=\lowenr] at ($( 3*\latWidth,3*\latWidth)$) {}; % E5
        \node [Assembly, fill=\midenr] at ($( 4*\latWidth,3*\latWidth)$) {}; % D5
        \node [Assembly, fill=\lowenr] at ($( 5*\latWidth,3*\latWidth)$) {$\mathrm{S}_\mathrm{D}$}; % C5
        \node [Assembly, fill=\highenr] at ($( 6*\latWidth,3*\latWidth)$) {}; % B5
        \node [Assembly, fill=\highenr] at ($( 7*\latWidth,3*\latWidth)$) {}; % A5

        \node [Assembly, fill=\highenr] at ($(-7*\latWidth,2*\latWidth)$) {}; % R6
        \node [Assembly, fill=\lowenr] at ($(-6*\latWidth,2*\latWidth)$) {$\mathrm{B}$}; % P6
        \node [Assembly, fill=\midenr] at ($(-5*\latWidth,2*\latWidth)$) {}; % N6
        \node [Assembly, fill=\lowenr] at ($(-4*\latWidth,2*\latWidth)$) {}; % M6
        \node [Assembly, fill=\midenr] at ($(-3*\latWidth,2*\latWidth)$) {}; % L6
        \node [Assembly, fill=\lowenr] at ($(-2*\latWidth,2*\latWidth)$) {$\mathrm{C}$}; % K6
        \node [Assembly, fill=\midenr] at ($(-1*\latWidth,2*\latWidth)$) {}; % J6
        \node [Assembly, fill=\lowenr] at ($(-0*\latWidth,2*\latWidth)$) {$\mathrm{A}$}; % H6
        \node [Assembly, fill=\midenr] at ($( 1*\latWidth,2*\latWidth)$) {}; % G6
        \node [Assembly, fill=\lowenr] at ($( 2*\latWidth,2*\latWidth)$) {$\mathrm{C}$}; % F6
        \node [Assembly, fill=\midenr] at ($( 3*\latWidth,2*\latWidth)$) {}; % E6
        \node [Assembly, fill=\lowenr] at ($( 4*\latWidth,2*\latWidth)$) {}; % D6
        \node [Assembly, fill=\midenr] at ($( 5*\latWidth,2*\latWidth)$) {}; % C6
        \node [Assembly, fill=\lowenr] at ($( 6*\latWidth,2*\latWidth)$) {$\mathrm{B}$}; % B6
        \node [Assembly, fill=\highenr] at ($( 7*\latWidth,2*\latWidth)$) {}; % A6

        \node [Assembly, fill=\highenr] at ($(-7*\latWidth,1*\latWidth)$) {}; % R7
        \node [Assembly, fill=\highenr] at ($(-6*\latWidth,1*\latWidth)$) {}; % P7
        \node [Assembly, fill=\lowenr] at ($(-5*\latWidth,1*\latWidth)$) {$\mathrm{S}_\mathrm{B}$}; % N7
        \node [Assembly, fill=\midenr] at ($(-4*\latWidth,1*\latWidth)$) {}; % M7
        \node [Assembly, fill=\lowenr] at ($(-3*\latWidth,1*\latWidth)$) {}; % L7
        \node [Assembly, fill=\midenr] at ($(-2*\latWidth,1*\latWidth)$) {}; % K7
        \node [Assembly, fill=\lowenr] at ($(-1*\latWidth,1*\latWidth)$) {}; % J7
        \node [Assembly, fill=\midenr] at ($(-0*\latWidth,1*\latWidth)$) {}; % H7
        \node [Assembly, fill=\lowenr] at ($( 1*\latWidth,1*\latWidth)$) {}; % G7
        \node [Assembly, fill=\midenr] at ($( 2*\latWidth,1*\latWidth)$) {}; % F7
        \node [Assembly, fill=\lowenr] at ($( 3*\latWidth,1*\latWidth)$) {}; % E7
        \node [Assembly, fill=\midenr] at ($( 4*\latWidth,1*\latWidth)$) {}; % D7
        \node [Assembly, fill=\lowenr] at ($( 5*\latWidth,1*\latWidth)$) {$\mathrm{S}_\mathrm{B}$}; % C7
        \node [Assembly, fill=\highenr] at ($( 6*\latWidth,1*\latWidth)$) {}; % B7
        \node [Assembly, fill=\highenr] at ($( 7*\latWidth,1*\latWidth)$) {}; % A7

        \node [Assembly, fill=\highenr] at ($(-7*\latWidth,0*\latWidth)$) {}; % R8
        \node [Assembly, fill=\lowenr] at ($(-6*\latWidth,0*\latWidth)$) {$\mathrm{C}$}; % P8
        \node [Assembly, fill=\midenr] at ($(-5*\latWidth,0*\latWidth)$) {}; % N8
        \node [Assembly, fill=\lowenr] at ($(-4*\latWidth,0*\latWidth)$) {$\mathrm{S}_\mathrm{E}$}; % M8
        \node [Assembly, fill=\midenr] at ($(-3*\latWidth,0*\latWidth)$) {}; % L8
        \node [Assembly, fill=\lowenr] at ($(-2*\latWidth,0*\latWidth)$) {$\mathrm{A}$}; % K8
        \node [Assembly, fill=\midenr] at ($(-1*\latWidth,0*\latWidth)$) {}; % J8
        \node [Assembly, fill=\lowenr] at ($(-0*\latWidth,0*\latWidth)$) {$\mathrm{D}$}; % H8
        \node [Assembly, fill=\midenr] at ($( 1*\latWidth,0*\latWidth)$) {}; % G8
        \node [Assembly, fill=\lowenr] at ($( 2*\latWidth,0*\latWidth)$) {$\mathrm{A}$}; % F8
        \node [Assembly, fill=\midenr] at ($( 3*\latWidth,0*\latWidth)$) {}; % E8
        \node [Assembly, fill=\lowenr] at ($( 4*\latWidth,0*\latWidth)$) {$\mathrm{S}_\mathrm{E}$}; % D8
        \node [Assembly, fill=\midenr] at ($( 5*\latWidth,0*\latWidth)$) {}; % C8
        \node [Assembly, fill=\lowenr] at ($( 6*\latWidth,0*\latWidth)$) {$\mathrm{C}$}; % B8
        \node [Assembly, fill=\highenr] at ($( 7*\latWidth,0*\latWidth)$) {}; % A8

        \node [Assembly, fill=\highenr] at ($(-7*\latWidth,-1*\latWidth)$) {}; % R9
        \node [Assembly, fill=\highenr] at ($(-6*\latWidth,-1*\latWidth)$) {}; % P9
        \node [Assembly, fill=\lowenr] at ($(-5*\latWidth,-1*\latWidth)$) {$\mathrm{S}_\mathrm{B}$}; % N9
        \node [Assembly, fill=\midenr] at ($(-4*\latWidth,-1*\latWidth)$) {}; % M9
        \node [Assembly, fill=\lowenr] at ($(-3*\latWidth,-1*\latWidth)$) {}; % L9
        \node [Assembly, fill=\midenr] at ($(-2*\latWidth,-1*\latWidth)$) {}; % K9
        \node [Assembly, fill=\lowenr] at ($(-1*\latWidth,-1*\latWidth)$) {}; % J9
        \node [Assembly, fill=\midenr] at ($(-0*\latWidth,-1*\latWidth)$) {}; % H9
        \node [Assembly, fill=\lowenr] at ($( 1*\latWidth,-1*\latWidth)$) {}; % G9
        \node [Assembly, fill=\midenr] at ($( 2*\latWidth,-1*\latWidth)$) {}; % F9
        \node [Assembly, fill=\lowenr] at ($( 3*\latWidth,-1*\latWidth)$) {}; % E9
        \node [Assembly, fill=\midenr] at ($( 4*\latWidth,-1*\latWidth)$) {}; % D9
        \node [Assembly, fill=\lowenr] at ($( 5*\latWidth,-1*\latWidth)$) {$\mathrm{S}_\mathrm{B}$}; % C9
        \node [Assembly, fill=\highenr] at ($( 6*\latWidth,-1*\latWidth)$) {}; % B9
        \node [Assembly, fill=\highenr] at ($( 7*\latWidth,-1*\latWidth)$) {}; % A9

        \node [Assembly, fill=\highenr] at ($(-7*\latWidth,-2*\latWidth)$) {}; % R10
        \node [Assembly, fill=\lowenr] at ($(-6*\latWidth,-2*\latWidth)$) {$\mathrm{B}$}; % P10
        \node [Assembly, fill=\midenr] at ($(-5*\latWidth,-2*\latWidth)$) {}; % N10
        \node [Assembly, fill=\lowenr] at ($(-4*\latWidth,-2*\latWidth)$) {}; % M10
        \node [Assembly, fill=\midenr] at ($(-3*\latWidth,-2*\latWidth)$) {}; % L10
        \node [Assembly, fill=\lowenr] at ($(-2*\latWidth,-2*\latWidth)$) {$\mathrm{C}$}; % K10
        \node [Assembly, fill=\midenr] at ($(-1*\latWidth,-2*\latWidth)$) {}; % J10
        \node [Assembly, fill=\lowenr] at ($(-0*\latWidth,-2*\latWidth)$) {$\mathrm{A}$}; % H10
        \node [Assembly, fill=\midenr] at ($( 1*\latWidth,-2*\latWidth)$) {}; % G10
        \node [Assembly, fill=\lowenr] at ($( 2*\latWidth,-2*\latWidth)$) {$\mathrm{C}$}; % F10
        \node [Assembly, fill=\midenr] at ($( 3*\latWidth,-2*\latWidth)$) {}; % E10
        \node [Assembly, fill=\lowenr] at ($( 4*\latWidth,-2*\latWidth)$) {}; % D10
        \node [Assembly, fill=\midenr] at ($( 5*\latWidth,-2*\latWidth)$) {}; % C10
        \node [Assembly, fill=\lowenr] at ($( 6*\latWidth,-2*\latWidth)$) {$\mathrm{B}$}; % B10
        \node [Assembly, fill=\highenr] at ($( 7*\latWidth,-2*\latWidth)$) {}; % A10

        \node [Assembly, fill=\highenr] at ($(-7*\latWidth,-3*\latWidth)$) {}; % R11
        \node [Assembly, fill=\highenr] at ($(-6*\latWidth,-3*\latWidth)$) {}; % P11
        \node [Assembly, fill=\lowenr] at ($(-5*\latWidth,-3*\latWidth)$) {$\mathrm{S}_\mathrm{D}$}; % N11
        \node [Assembly, fill=\midenr] at ($(-4*\latWidth,-3*\latWidth)$) {}; % M11
        \node [Assembly, fill=\lowenr] at ($(-3*\latWidth,-3*\latWidth)$) {}; % L11
        \node [Assembly, fill=\midenr] at ($(-2*\latWidth,-3*\latWidth)$) {}; % K11
        \node [Assembly, fill=\lowenr] at ($(-1*\latWidth,-3*\latWidth)$) {}; % J11
        \node [Assembly, fill=\midenr] at ($(-0*\latWidth,-3*\latWidth)$) {}; % H11
        \node [Assembly, fill=\lowenr] at ($( 1*\latWidth,-3*\latWidth)$) {}; % G11
        \node [Assembly, fill=\midenr] at ($( 2*\latWidth,-3*\latWidth)$) {}; % F11
        \node [Assembly, fill=\lowenr] at ($( 3*\latWidth,-3*\latWidth)$) {}; % E11
        \node [Assembly, fill=\midenr] at ($( 4*\latWidth,-3*\latWidth)$) {}; % D11
        \node [Assembly, fill=\lowenr] at ($( 5*\latWidth,-3*\latWidth)$) {$\mathrm{S}_\mathrm{C}$}; % C11
        \node [Assembly, fill=\highenr] at ($( 6*\latWidth,-3*\latWidth)$) {}; % B11
        \node [Assembly, fill=\highenr] at ($( 7*\latWidth,-3*\latWidth)$) {}; % A11

        \node [Assembly, fill=\highenr] at ($(-6*\latWidth,-4*\latWidth)$) {$\mathrm{S}_\mathrm{A}$}; % P12
        \node [Assembly, fill=\midenr] at ($(-5*\latWidth,-4*\latWidth)$) {}; % N12
        \node [Assembly, fill=\midenr] at ($(-4*\latWidth,-4*\latWidth)$) {$\mathrm{D}$}; % M12
        \node [Assembly, fill=\midenr] at ($(-3*\latWidth,-4*\latWidth)$) {}; % L12
        \node [Assembly, fill=\lowenr] at ($(-2*\latWidth,-4*\latWidth)$) {}; % K12
        \node [Assembly, fill=\midenr] at ($(-1*\latWidth,-4*\latWidth)$) {}; % J12
        \node [Assembly, fill=\lowenr] at ($(-0*\latWidth,-4*\latWidth)$) {$\mathrm{S}_\mathrm{E}$}; % H12
        \node [Assembly, fill=\midenr] at ($( 1*\latWidth,-4*\latWidth)$) {}; % G12
        \node [Assembly, fill=\lowenr] at ($( 2*\latWidth,-4*\latWidth)$) {}; % F12
        \node [Assembly, fill=\midenr] at ($( 3*\latWidth,-4*\latWidth)$) {}; % E12
        \node [Assembly, fill=\midenr] at ($( 4*\latWidth,-4*\latWidth)$) {$\mathrm{D}$}; % D12
        \node [Assembly, fill=\midenr] at ($( 5*\latWidth,-4*\latWidth)$) {}; % C12
        \node [Assembly, fill=\highenr] at ($( 6*\latWidth,-4*\latWidth)$) {$\mathrm{S}_\mathrm{A}$}; % B12

        \node [Assembly, fill=\highenr] at ($(-6*\latWidth,-5*\latWidth)$) {}; % P13
        \node [Assembly, fill=\highenr] at ($(-5*\latWidth,-5*\latWidth)$) {}; % N13
        \node [Assembly, fill=\midenr] at ($(-4*\latWidth,-5*\latWidth)$) {}; % M13
        \node [Assembly, fill=\lowenr] at ($(-3*\latWidth,-5*\latWidth)$) {$\mathrm{S}_\mathrm{C}$}; % L13
        \node [Assembly, fill=\midenr] at ($(-2*\latWidth,-5*\latWidth)$) {}; % K13
        \node [Assembly, fill=\lowenr] at ($(-1*\latWidth,-5*\latWidth)$) {$\mathrm{S}_\mathrm{B}$}; % J13
        \node [Assembly, fill=\midenr] at ($(-0*\latWidth,-5*\latWidth)$) {}; % H13
        \node [Assembly, fill=\lowenr] at ($( 1*\latWidth,-5*\latWidth)$) {$\mathrm{S}_\mathrm{B}$}; % G13
        \node [Assembly, fill=\midenr] at ($( 2*\latWidth,-5*\latWidth)$) {}; % F13
        \node [Assembly, fill=\lowenr] at ($( 3*\latWidth,-5*\latWidth)$) {$\mathrm{S}_\mathrm{D}$}; % E13
        \node [Assembly, fill=\midenr] at ($( 4*\latWidth,-5*\latWidth)$) {}; % D13
        \node [Assembly, fill=\highenr] at ($( 5*\latWidth,-5*\latWidth)$) {}; % C13
        \node [Assembly, fill=\highenr] at ($( 6*\latWidth,-5*\latWidth)$) {}; % B13

        \node [Assembly, fill=\highenr] at ($(-5*\latWidth,-6*\latWidth)$) {}; % N14
        \node [Assembly, fill=\highenr] at ($(-4*\latWidth,-6*\latWidth)$) {$\mathrm{S}_\mathrm{A}$}; % M14
        \node [Assembly, fill=\highenr] at ($(-3*\latWidth,-6*\latWidth)$) {}; % L14
        \node [Assembly, fill=\lowenr] at ($(-2*\latWidth,-6*\latWidth)$) {$\mathrm{B}$}; % K14
        \node [Assembly, fill=\highenr] at ($(-1*\latWidth,-6*\latWidth)$) {}; % J14
        \node [Assembly, fill=\lowenr] at ($(-0*\latWidth,-6*\latWidth)$) {$\mathrm{C}$}; % H14
        \node [Assembly, fill=\highenr] at ($( 1*\latWidth,-6*\latWidth)$) {}; % G14
        \node [Assembly, fill=\lowenr] at ($( 2*\latWidth,-6*\latWidth)$) {$\mathrm{B}$}; % F14
        \node [Assembly, fill=\highenr] at ($( 3*\latWidth,-6*\latWidth)$) {}; % E14
        \node [Assembly, fill=\highenr] at ($( 4*\latWidth,-6*\latWidth)$) {$\mathrm{S}_\mathrm{A}$}; % D14
        \node [Assembly, fill=\highenr] at ($( 5*\latWidth,-6*\latWidth)$) {}; % C14

        \node [Assembly, fill=\highenr] at ($(-3*\latWidth,-7*\latWidth)$) {}; % L15
        \node [Assembly, fill=\highenr] at ($(-2*\latWidth,-7*\latWidth)$) {}; % K15
        \node [Assembly, fill=\highenr] at ($(-1*\latWidth,-7*\latWidth)$) {}; % J15
        \node [Assembly, fill=\highenr] at ($(-0*\latWidth,-7*\latWidth)$) {}; % H15
        \node [Assembly, fill=\highenr] at ($( 1*\latWidth,-7*\latWidth)$) {}; % G15
        \node [Assembly, fill=\highenr] at ($( 2*\latWidth,-7*\latWidth)$) {}; % F15
        \node [Assembly, fill=\highenr] at ($( 3*\latWidth,-7*\latWidth)$) {}; % E15

      \end{tikzpicture}
    }
    


    \caption[Control rod and shutdown bank positions.]{Control rod and shutdown bank positions. Source: \ref{num:assycore} \label{fig_cr_pos}}
\end{figure}

 % label: fig_cr_pos

%%%%%%%%%%%%%%%%%%%%%%%%%%%%%%%%%%%%%%%%%%%%%%%%%%%%%%%%%%%%%%%%%%%%%%%%%%%%%%%%
\FloatBarrier
\paragraph{Instrument Tube Positions}
\label{sec:coreinstrpos}

The central guide tube for many fuel assemblies in the core is filled by an
instrument tube, as described in Section \ref{sec:pintypes}. Figure
\ref{fig_instr_pos} shows these positions. Where not indicated, the central
guide tube is filled with water, as described in section \ref{sec:pintypes}.

The positions of instrument tubes do not change between cycle 1 and cycle 2.

\input{specifications/core/figs/cat_instr_pos} % label: fig_instr_pos

\FloatBarrier
